%%
%% 研究報告用スイッチ
%% [techrep]
%%
%% 欧文表記無しのスイッチ(etitle,jkeyword,eabstract,ekeywordは任意)
%% [noauthor]
%%

\documentclass[submit,techrep]{ipsj}
%\documentclass[submit,techrep,noauthor]{ipsj}

\usepackage[dvipdfm]{graphicx}
\usepackage{latexsym}
\usepackage{url}

\def\Underline{\setbox0\hbox\bgroup\let\\\endUnderline}
\def\endUnderline{\vphantom{y}\egroup\smash{\underline{\box0}}\\}
\def\|{\verb|}

\setcounter{巻数}{53}%vol53=2012
\setcounter{号数}{10}
\setcounter{page}{1}


\begin{document}

\paffiliate{JU}{愛知工業大学\\
AICHI INSTITUTE OF TECHNOLOGY University}

\title{データベース及び演習 期末レポート\\}

\author{K20089 西宮 銀河}{K20089 Nishimiya Ginga}{IPSJ, JU}[gingin.sol@pluslab.org]

\begin{abstract}
本稿は, データベースおよび演習の授業における最終課題についてのレポートである. 
今回はかんたん備忘録という共有可能なTODOリストを作成した. 機能の詳細は以下より述べる.
\end{abstract}

\maketitle

%1
\section{機能概要}



%2
\section{利用技術}

%2.1
%この節における参考文献
%https://www.php.net/manual/ja/faq.general.php
%http://php.adamharvey.name/manual/ja/internals2.ze1.zendapi.php
%https://www.phptutorial.net/php-tutorial/what-is-php/
\subsection{PHP}
PHPとは"PHP:Hypertext Preprocessor"の略で, 直接HTMLに埋め込んで使用することができる汎用プログラミング言語である. 
構文はC言語, Java, Perlから派生しており, それに加えてPHP独自の構文を組み込むことでPHP特有の機能を使用できる. 
PHPはサーバサイド言語であるので, 主にWebサーバ側で動作する. 
なのでクライアント側からブラウザがHTTPリクエストを送ったとすると, サーバ側がPHPを処理して生成されたHTML文書などをクライアント側に返すという仕組みとなっている. 
そしてPHPはクロスプラットフォーム言語なのでLinux, Windows, MacOSはもちろん, NginXやApacheなど全てのwebサーバ, さらにMicrosoft AzureやAmazon AWSなどのクラウド環境にも対応している. 

また現在の最新バージョンはPHP 8であり, これはZend Engine 4というPHPの実行エンジンの最新バージョンを使用している. これによってオブジェクト指向プログラミングの機能が使えるようになっている. 

%2.2
%この節における参考文献
%https://kinsta.com/jp/knowledgebase/what-is-mysql/
%https://mariadb.org/
%https://www.integrate.io/jp/blog/mariadb-vs-mysql-everything-you-need-to-know-ja/#what
\subsection{MariaDB/MySQL}
まず, MySQLは1995年に開発されたオープンソースソフトウェアのリレーショナルデータベースである. 
'95年以降は所有者も管理者も転々と変わっていたが2010年に所有権がOracle社に移るが, 所有権が渡った今でもオープンソースであることは変わっていない.

MySQLはリレーショナルデータベースと呼ばれるデータの保存方法が使われている. リレーショナルデータベースとは1つのテーブルに全ての情報を挿入するのではなく, 複数のテーブルに分けてそれぞれを関連づけて管理する方法である. 
例えば名前や住所などの顧客データと購入製品や価格などの注文データを扱うテーブルを作るとすると, もし一つのテーブルに全てのデータを入れておくとなった際にデータの重複やデータ削除時の混乱が起こり, 検索が難しくなる. その問題に対処するためにリレーショナルデータベースでは
キーと呼ばれるユニークなIDを利用して異なるテーブル同士を繋げ, データの取り出しや削除などが容易になっている. 

そして, MariaDBはこのMySQLをベースに作られたデータベースである. MySQLと比べてパフォーマンス, 安定性, 開示性が高く, もっとも有名なリレーショナルデータベースの一つとされている. MySQLとの完全な互換性, GPLの採用, さらには開発者コミュニティが豊富なのでMariaDBはMySQLの上位互換と言える. 
%2.3
%この節における参考文献
%https://developer.mozilla.org/ja/docs/Web/CSS
\subsection{CSS}
CSSとは"Cascading Style Sheets"の略で, HTMLやXML文書の見栄えなどを整えるために使われるスタイルシート言語である. 
HTML内で指定されたタグやidなどをセレクタで指定して色や段落などの設定ができる. 

%2.4
%この節における参考文献
%https://www.smarty.net/docsv2/ja/what.is.smarty.tpl
\subsection{Smarty}
SmartyとはPHP用のテンプレートエンジンである. 図1に示すようにプロジェクトファイルをデータベースと接続するモデル, HTML表示を行うためにクライアントにレスポンスを返すビュー, これらモデルとビューをつなぐコントローラの3つの役割に分けることでフロントエンドとバックエンドそれぞれの開発がしやすくなっている. デリミタの使用ができたり, 構文はPHPパーサが処理するので複雑な条件式も入力でき, さらに高速で動作できることが特徴となっている.



%https://developer.mozilla.org/ja/docs/Web/HTTP
\subsection{HTTP}
HTTPとは"Hypertext Transfer Protocol"の略でハイパーテキストを転送するためのアプリケーション層プロトコルである. 
主にサーバ側とクライアント側との通信の目的のために使われるが別の用途でも使用されることがある. 基本的な動作としてはクライアント側がサーバ側にリクエストを送信するためにポートを開き, サーバ側からレスポンスが帰ってくるまで待機するという仕組みとなっている. 
また, HTTPは主にTCP/IP層の上の通信で使用されるが, UDPのようなトランスポート層でも使用されることがある. 

%https://www.siteground.com/tutorials/email/protocols-pop3-smtp-imap/
%https://www.siteground.com/kb/what-is-smtp/
\subsection{SMTP/IMAP/POP}
SMTPは"Simple Mail Transfer Protocol"の略で, メール送信の際に用いられるプロトコルである.
このプロトコルでスパムメールを防ぐためにはSMTP認証が必要である. そのためにはメールアドレスとパスワードの入力でメールサーバーに接続することができ, 安全なメール送信が可能となる. 

IMAPは"Internet Message Access Protocol"の略で, メール受信の際に用いられるプロトコルである.
クライアント側からリモートウェブサーバにアクセスしてメールを確認することができるので, 複数の端末から同時に接続できたり, 端末が変わっても接続することができる.

POPは"Post Office Protocol"の略で, こちらもIMAP同様, メール送信の際に用いられるプロトコルである. 
IMAPとの違いはリモートサーバ側からクライアントに接続してメールをローカルにダウンロードするのでオフラインでもメールの確認ができるようになっている. さらにダウンロードした後はサーバに残っているメッセージは削除されるのでサーバの容量の削減になることが特徴となっている.

%3
\section{システム設計}
\subsection{システム概要}
Smartyを使って, コントローラでデータベースと表示を操作している. 実際会員ページではMemberController.php, 管理者ページではSystemController.phpを使ってモデルから受け取ったデータベースの情報をビューに送り, 表示させている.  
中で使用されているコントローラは全てBaseController.phpというベースとなるコントローラを継承しており, BaseControllerのログイン状態やエラー表示の機能に加えて, それぞれのコントローラで使用される機能を追加している. 

モデルも同様にBaseModel.phpというベースファイルを継承してそれぞれの場面で使用されるモデルクラスを生成している. 
このクラスではデータベースに接続して様々な操作をする機能を持っており, このプロジェクトでは新規登録時のデータベースへの挿入や退会時のデータベースからの削除, 管理者画面でのデータの検索などをする際にモデルの役目が果たされていると言える. 

また, ビューで表示するHTML文書は拡張子が.htmlではなく, .tplというテンプレートファイルを使って表示している. これを使えばif構文などが使用できるのでコントローラから受け取った値などをhtmlに便利に表示することができる. 

\subsection{画面遷移}
まず,  会員ページでの状態遷移図は図2のような構成となっている.  ログイン画面からは新規登録画面とログインした場合の会員のトップ画面に飛ぶことができる. 
新規登録画面ではメールアドレスなど必要な情報を入力してもらい, その次に登録情報の確認に移る. そしてメールの送信が完了し, 登録したメールアドレスに送られた専用メールに書かれたリンクに飛ぶと, 登録完了となり, ログイン画面に戻ることができる. 
この際にすでに有効期限などが切れ, 無効となったリンクに飛ぼうとするとエラーとなり, 同じくログイン画面に戻ることになる. 

会員登録が済んでいる場合はログイン画面で会員情報を入力して会員トップ画面に飛ぶことができる. 会員トップ画面では登録情報の変更と退会をすることができ, またトップページからログアウトをすることも可能である. 
更新画面では新規登録時と同じ入力フォームが表示され, データの変更ができる. 次に確認画面で確認を行い更新完了となる. 
そして退会画面では本当に退会してもいいかの削除確認画面が表示され, よければ削除完了となりログアウトされ, ログイン画面に戻る. 

次は管理者ページでの状態遷移図は図3のようになっている. まず, 管理者用のユーザ名とパスワードを入力すると管理トップ画面に行くことができる. 管理トップ画面からは会員一覧の表示に飛ぶことができ, そこでは会員の新規登録, 更新, 削除を行うことができる. 

新規登録を行う際はまず登録情報の入力, そして確認画面にて確認をすることで登録を完了することができる. 更新画面でも同様に登録情報の変更を行い, 確認をして更新が完了される. 削除画面では本当に削除していいかの表示の確認の後, 会員の削除を行うことできる. 

\subsection{データベース設計}
本システムで使用されているデータベースでは次の5つのテーブルが使われている. 
\begin{quote}
 \begin{itemize}
  \item 会員情報テーブル
  \item 県名テーブル
  \item 仮登録テーブル
  \item 管理者情報テーブル
  \item お知らせテーブル
 \end{itemize}
\end{quote}
会員情報テーブルでは本登録された会員情報を保持し, 県名テーブルでは会員情報入力に必要な都道府県名を保持している.
仮登録テーブルでは仮登録の際にメールを送信したアカウント情報が保持されている. 
このテーブルでは本登録が完了すれば該当データは自動的に削除されるようになっている. そして管理者情報テーブルでは会員ではない管理者の情報を保持するのに使用されている. 管理者ページにログインする際はこちらのテーブルから参照される.
最後にお知らせテーブルはお知らせの文章を保持するために使われている.  

会員情報テーブルのテーブル構成は表1のようになっている. 一意の数値を挿入するためのidフィールドがまず最初にあることが分かる. 
このフィールドは3バイトの整数型を取扱い, 数値が無であることは許されておらず, さらに自動繰上げ機能が付いている. これによってデータが挿入されると自動的にidの値を繰り上げることが可能となっている. 
次に続くusernameフィールドからfirst\_nameフィールドに関してはユーザネーム, パスワード, 名字, 名前を示しており, 括弧内の文字数分を確保した文字列を格納している. birthdayフィールドは誕生日を示し, 8桁の文字型で格納されている. kenフィールドでは2バイトの整数型を取り扱い, 県名テーブルと連携して都道府県のそれぞれのidが入ることになっている. 
reg\_dateフィールドでは日付型を扱っており, 登録日時を格納している. 
\begin{table}[htb]
\centering
  \caption{会員情報テーブル構成}
  \scalebox{0.7}{
  \begin{tabular}{|l|c|r|c|c|c|}  \hline
    	Field & Type & Null & Key & Default & Extra \\ \hline \hline
   	 id & mediumint unsigned & NO & PRIMARY & NULL & auto\_increment \\ \hline
	 username & varchar(50) & YES &  & NULL & \\ \hline
	 password & varchar(128) & YES & & NULL & \\ \hline
	 last\_name & varchar(50) & YES & & NULL & \\ \hline
	 first\_name & varchar(50) & YES & & NULL & \\ \hline
	 birthday & char(8) & YES &  & NULL & \\ \hline
	 ken & smallint & YES &  & NULL & \\ \hline
	 reg\_date & datetime & YES & & NULL & \\ \hline
	 cancel & datetime & YES & & NULL & \\ \hline
   	\end{tabular}
   }
\end{table}

次に県名テーブルである. 県名テーブルの構成は表2の通りである. 同じく一意の数値が格納されるidフィールドがあるが, 今回は47都道府県分の番号を入れるのみなのでsmallint型を使用して2バイトの数値を扱うようにしている. 
また, kenフィールドでは10文字の都道部県名が格納されるようになっている. 

\begin{table}[htb]
\centering
  \caption{県名テーブル構成}
  \scalebox{0.7}{
  \begin{tabular}{|l|c|r|c|c|c|}  \hline
    	Field & Type & Null & Key & Default & Extra \\ \hline \hline
   	 id & smallint & NO & PRIMARY & NULL &  \\ \hline
	 ken & varchar(10) & YES & & & \\ \hline
   	\end{tabular}
   }
\end{table}

次は仮登録テーブルである. 仮登録テーブルの構成は表3に示す通りである. このテーブルの構成はほとんど会員情報テーブルとほぼ同じだが, 仮登録完了メールに書かれたリンクに飛ぶ際に使用されるパスワードを格納するlink\_passフィールドが追加されている. このパスワードを使用して本登録が完了したかなどの判別を行なっている. 

\begin{table}[htb]
\centering
  \caption{仮登録テーブル構成}
  \scalebox{0.7}{
  \begin{tabular}{|l|c|r|c|c|c|}  \hline
    	Field & Type & Null & Key & Default & Extra \\ \hline \hline
   	 id & mediumint unsigned & NO & PRIMARY & NULL & auto\_increment \\ \hline
	 username & varchar(50) & YES &  & NULL & \\ \hline
	 password & varchar(128) & YES & & NULL & \\ \hline
	 last\_name & varchar(50) & YES & & NULL & \\ \hline
	 first\_name & varchar(50) & YES & & NULL & \\ \hline
	 birthday & char(8) & YES &  & NULL & \\ \hline
	 ken & smallint & YES &  & NULL & \\ \hline
	 link\_pass & varchar(128) & YES &  & NULL & \\ \hline
	 reg\_date & datetime & YES & & NULL & \\ \hline
   	\end{tabular}
   }
\end{table}

最後にお知らせテーブルである. お知らせテーブルの構成は表4に示すものである. 今までと同じように一意なidを最初にフィールドとして定義している. 次にお知らせの件名であるsubjectフィールドを256文字までという制限とともに定義している. 
また本文であるbodyフィールドにはtext型を宣言しているので自由な文字列が入力と可能となっており, 最後にはお知らせがされた時間を挿入するreg\_dateが定義されている. 

\begin{table}[htb]
\centering
  \caption{お知らせテーブル構成}
  \scalebox{0.7}{
  \begin{tabular}{|l|c|r|c|c|c|}  \hline
    	Field & Type & Null & Key & Default & Extra \\ \hline \hline
   	 id & mediumint unsigned & NO & PRIMARY & NULL & auto\_increment \\ \hline
	 subject & varchar(256) & YES &  & NULL & \\ \hline
	 body & text & YES & & NULL & \\ \hline
	 reg\_date & datetime & YES & & NULL & \\ \hline
   	\end{tabular}
   }
\end{table}


\subsection{システム詳細}
会員ページのメインファイルとなるindex.phpではまず, 初期化ファイルのinit.phpを呼び出し, データベースのユーザ設定やライブラリなどの初期設定を行なっている. 
次にMemberController.phpという会員情報を取り扱うコントローラクラスを実体化して, 更にAuthクラスの機能を使用してログインされている状態かを判別することで表示するページを変えている. MemberControllerで使われるページ遷移の状態には次のような6つの種類がある. ちなみにこれらのステートメントはURLの末尾につけられるので状態の判別ができるようになっている. 
\begin{quote}
 \begin{itemize}
  \item ログイン時
   \begin{itemize}
    \item デフォルト状態 : 会員トップページ
    \item "logout" : ログアウトされたとき
    \item "modify" : 登録情報が変更されたとき
    \item "delete" : 退会するとき
   \end{itemize}
 \item 未ログイン時
   \begin{itemize}
    \item デフォルト状態 : ログインページ
    \item "regist" : 新規登録するとき
    \item "authenticate" : 登録データが条件に合っているか判別するとき
   \end{itemize}
 \end{itemize}
\end{quote}


またこのコントローラはBaseControllerというクラスを継承しており, このクラスではSmartyの実装や会員情報入力時のルールなどについて定義している. このプロジェクトで使用されるコントローラは全てこのベースコントローラを継承している.

そしてNoticeControllerではお知らせに関するデータを扱っている. まずNoticeModelクラスを使用してお知らせデータベースからデータを持ってくることでお知らせの文章を取り出すことができる. 
あとはテンプレートファイルで整った表示をすることでお知らせをデータベース上のお知らせ情報を表示させることが可能となっている.  

次のPrememberControllerは仮登録メールから本登録を行う際に用いられている. 
PrememberModelという仮登録データベースを扱うモデルクラスを使用して, リンクに記載されているリンクパスワードの一致とユーザの名前が一致していれば本登録完了の文を表示し, 一致しなかったりリンクにパラメータがなければエラーを表示することで本登録処理を行なっている.  

最後にSystemControllerである. これは管理者画面で使用されるコントローラである. Authクラスを使用してセッション開始と認証を行い, まだ認証がされていない場合は状態を変えてログインページに飛ぶようになっている. 
また, MemberController同様, ページ遷移のためのパラメータを扱っており, 次の9つのステートメントがある.

\begin{quote}
 \begin{itemize}
 \item デフォルト状態 : ログインページまたは管理者ページ
 \item "login" : ログインページにいるとき
 \item "logout" : ログアウトされたとき
 \item "modify" : 会員情報を変更するとき
 \item "delete" : 会員情報を削除するとき
 \item "list" : 会員情報を表示しているとき
 \item "regist" : 会員情報を新規で登録するとき
 \item "notice" : お知らせ表示を編集するとき
 \item "authenticate" : 登録データが条件に合っているか判別するとき
 \end{itemize}
\end{quote}
 
 これらの状態に応じて表示するテンプレートファイルを切り替えている. 

\subsection{変数設計}
本システムで使われる変数の多くは主にBaseControllerで宣言されている. まずtype変数とaction変数はそれぞれ状態遷移のときに使われている. type変数はそれぞれのページにおける"login"や"modify"などの状態の名前を保持する役割を持っている. 
action変数は何かしらの動作を行おうとしている時にその動作の名前を保持するものである. この場合の例ではフォーム入力するときの"form", 確認画面であることを示す"confirm", 動作が完了したことを示す"complete"などがある. 
次のnext\_type変数とnext\_action変数は名前の通り次に遷移する状態を保持する. 

file変数では使うテンプレートファイル名を格納している. ここにファイルを定義することでHTMLを表示している. 

次のform変数ではフォーム入力する際に用いられるHTML要素が格納されている. ここにtextフォームやsubmitフォームを追加することでそれに応じたフォーム要素が使用できるようになる. このときHTML\_QuickForm2という入力フォームを生成するクラスが使用されているのでSmartyと組み合わせてテンプレート1枚で入力フォームが便利に生成可能となっている. 
この際にrenderer変数にHTML\_QuickForm2\_Rendererというクラスを割り当ててSmartyとの組み合わせが可能となっている.  また, auth変数にはAuthクラスを定義して認証機能が使用できるようになっており, さらにview変数にはSmartyクラスを定義することでSmartyの機能が使用できる. 

次のtitle変数では各ページの見出し表示, message変数では内容文を格納し, auth\_error\_mess変数には認証失敗時の表示をし, debug\_str変数ではデバッグ用の表示をしている. 最後のlogin\_state変数では会員でログインされているか, 管理者でログインされているかの状態を保持している. 

\section{実装}
\subsection{実装環境}
本システムの実装環境は以下の通りである. 

\begin{quote}
 \begin{itemize}
 \item XAMPP 7.4.28
 \item MariaDB 10.4.21
 \item PHP 8.1.4
 \end{itemize}
\end{quote}

\subsection{環境設定}
使用しているXAMPPはデフォルト設定なので使用しているポート番号は80であり, サーバネームは"localhost"という設定となっている. また, データベース設定としてはデータベース接続用のユーザネームは"root", パスワードは未設定, ホスト名は"localhost", データベース名は"memberdb"となっている. 

\subsection{動作検証}
新規登録までの画面遷移は図4のようになっている. 3.2でも述べたようにログイン画面から新規登録画面に飛ぶことで必要情報を入力して, 入力情報を確認した後に仮登録完了画面に飛ぶようになっている. 


次に図5は会員登録後, ログインを完了させるまでの画面遷移である. 先ほど登録した情報をフォームに入力するとログインが完了となり, 登録した名前が表示されるようになる.


最後に管理者画面でのログインの画面遷移を図6に示している. まず管理者用のユーザネームとパスワードを入力すると管理者トップページに飛ぶことができる. さらにそこから会員一覧画面に飛ぶことができ, そこでは登録情報の新規登録, 変更, 削除, 検索ができるようになっている. 

\section{まとめ}
今回は教科書9章・10章のサンプルプログラムについてまとめた. 普段フレームワークを使用する際はテンプレートを編集して開発することが多かったので, Smartyを使用したMVC方式の設計を全て自分で書くことはあまりなく, どういった形式でこのプログラムが書かれているかをしっかり理解することができた. 
他のMVC方式のフレームワークを使用する際にも今回学んだ知識を活かして開発を進めていきたいと感じた. 



\begin{thebibliography}{9}
\bibitem{php} PHP Group . "一般的な情報" . php. https://www.php.net/manual/ja/faq.general.php . (閲覧日:2022/05/29)
\bibitem{php} PHP Group . "Zend API: PHP のコアをハックする" . php . http://php.adamharvey.name/manual/ja/internals2.ze1.zendapi.php . (閲覧日:2022/05/29)
\bibitem{php} PHP TUTORIAL . "What is PHP" . phptutorial.net . https://www.phptutorial.net/php-tutorial/what-is-php/ (閲覧日:2022/05/29)
\bibitem{mysql} KINSTA . "MySQLとは?初心者にわかりやすい説明" . KINSTA . (更新日:2020/07/03) . https://kinsta.com/jp/knowledgebase/what-is-mysql/ . (閲覧日:2022/06/01)
\bibitem{mariaDB} MariaDB Foundation . "MariaDB Server: The open source relational database" . MariaDB Foundation . https://mariadb.org/ . (閲覧日:2022/06/01)
\bibitem{mariaDB} Mark Smallcombe . "MariaDB vs MySQL: 徹底比較" . integrate.io . (更新日:2020/09/03) . https://www.integrate.io/jp/blog/mariadb-vs-mysql-everything-you-need-to-know-ja/ . (閲覧日:2022/06/01)
\bibitem{css} MDN contributors . "CSS: カスケーディングスタイルシート" . mdn web docs . (更新日:2021/07/18) . https://developer.mozilla.org/ja/docs/Web/CSS . (閲覧日:2022/06/01)
\bibitem{smarty} smarty . "Smarty とは?" .smarty . https://www.smarty.net/docsv2/ja/what.is.smarty.tpl . (閲覧日:2022/06/01)
\bibitem{http} MDN contributors . "HTTP" . mdn web docs . (更新日:2021/09/18) . https://developer.mozilla.org/ja/docs/Web/HTTP . (閲覧日:2022/06/01)
\bibitem{smtp} SiteGround Hosting Ltd . "Email Protocols - POP3, SMTP and IMAP Tutorial" . SiteGround . https://www.siteground.com/tutorials/email/protocols-pop3-smtp-imap/ . (閲覧日:2022/06/01)
\bibitem{smtp} SiteGround Hosting Ltd . "What is SMTP?" . SiteGround . https://www.siteground.com/kb/what-is-smtp/. (閲覧日:2022/06/01)
\end{thebibliography}


\end{document}
